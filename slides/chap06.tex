\section{Complexity Theory}

\begin{frame}
  Most of the algorithms we explored in this course
  have been \emph{polynomial-time algorithms} (aka easy problems)\pause---that is,
  their worst-case running time is $\mathcal{O}(n^k)$.

    \begin{itemize}
     \item There are (hard) problems that can be solved, but not in time $\mathcal{O}(n^k)$ (for any constant $k$). 
     \item There are problems that cannot be solved by any computer (e.g., {\color{blue}The Halting Problem}), no metter
       how much computing time is available. 
    \end{itemize}
\end{frame}

\begin{frame}{P class of problems}
  \begin{itemize}
    \item The class P comprehends problems that we can solve in
    polynomial time ($\mathcal{O}(n^k)$).
  \end{itemize}    
\end{frame}


\begin{frame}{NP class of problems}
  \begin{itemize}
  \item The class NP comprehends problems that are verifiable
    in polynomial time. \pause That is, we are able to {\color{blue}check
    in polynomial time} the correctness of an answer for a
    specific instance of the problem.
  \item Example: we can check in polynomial time if an assignment to boolean variables
    satisfies a 3-CNF formula. However, we cannot answer if a 3-CNF formula is
    satisfiable in polynomial time. 
  \end{itemize}
\end{frame}  

\begin{frame}
  \begin{huge}
    Any problem in P is also in NP ($P \subseteq NP$).
  \end{huge} \pause

  \vskip+1.5em

  
  \begin{itemize}
   \item Open question: $P \subsetneq NP$
   \item NPC (NP-complete) is the set of problems that are in NP and
     are as ``hard'' as any problem in NP. \pause If an NPC problem could
     be solved in polynomial time, then all problems in NP has a polynomial
     time algorithm. \pause Most theoretical computer scientists believe that NPC
     problems are intractable.
  \end{itemize}
\end{frame}
